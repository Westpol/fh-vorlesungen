\documentclass[11pt, a4paper]{article}

	\author{Benno Schörmann}
% Übersicht:
%	1. Allgemeine Einstellungen
%	2. Nützliche Pakete
%	3. Neue Befehle
%	4. Gestaltung des Blattes

%%%%%%%%%%%%%%%%%%%%%%%%%%%%%%%%%%%%%%%%%%%%%%%%%%%%%%%%%%%%%%%%%%%%%%%%%%%%%%
%	A L L G E M E I N E  E I N S T E L L U N G E N
%%%%%%%%%%%%%%%%%%%%%%%%%%%%%%%%%%%%%%%%%%%%%%%%%%%%%%%%%%%%%%%%%%%%%%%%%%%%%%

% Spracheinstellungen und Kodierung
	\usepackage[ngerman]{babel}	% richtige Sprach- und Silbentrennung
	\usepackage[utf8]{inputenc}	% Eingabecodierung der Quelldatei
	\usepackage[T1]{fontenc}	% Zeichenkodierung der verwendeten Schriftarten
	
% Seitenlayout
	\usepackage[inner=2.5cm,outer=2.5cm,top=2cm,bottom=2cm]{geometry}

%%%%%%%%%%%%%%%%%%%%%%%%%%%%%%%%%%%%%%%%%%%%%%%%%%%%%%%%%%%%%%%%%%%%%%%%%%%%%%
%	N Ü T Z L I C H E  P A K E T E
%%%%%%%%%%%%%%%%%%%%%%%%%%%%%%%%%%%%%%%%%%%%%%%%%%%%%%%%%%%%%%%%%%%%%%%%%%%%%%

% Mathekram
	\usepackage{amsmath}	% Umgebungen und Funktionen für Gleichungsdarstellung
	\usepackage{amsthm}		% mathematische Umgebungen, wie Theorem, Sätze, Korollare,
	\usepackage{amsfonts}	% mathematische Schriftarten, z.B. für Integer oder Reals
	\usepackage{amssymb}	% weitere mathematische Symbole
	
% Tikz
	\usepackage{tikz}
	\usetikzlibrary{
		arrows,
		arrows.meta,
		automata,
		calc,
		decorations.pathmorphing,
		fadings,
		fit,
		positioning,
		shapes
	}
	\usepackage{pgfplots}

% Verweise
	\usepackage{nameref}

% Links
	\usepackage{hyperref}	% ermöglicht interne und externe Links
	\hypersetup{%
		colorlinks=true,
		%frenchlinks=true,
		linkcolor={black},
		%linkbordercolor = white,
	    citecolor={black},
	    urlcolor={myblue},
	    hyperindex=true
	}

% Farben
	\usepackage{xcolor}		% Verwaltung von Farben
	
% Aufzählungen
	\usepackage{enumitem}
	
% Tabellen
	\usepackage{array}
	\usepackage{multirow}
	
% Bedingte Anweisungen und Logik
	\usepackage{ifthen}
	\usepackage{pgffor}		% Schleifen
	
% Programmcodeumgebung
	\usepackage{listings}
	\lstset{language=Java}
	% Einstellungen beim Punkt "Gestaltung des Blattes"

%%%%%%%%%%%%%%%%%%%%%%%%%%%%%%%%%%%%%%%%%%%%%%%%%%%%%%%%%%%%%%%%%%%%%%%%%%%%%%
%	N E U E  B E F E H L E
%%%%%%%%%%%%%%%%%%%%%%%%%%%%%%%%%%%%%%%%%%%%%%%%%%%%%%%%%%%%%%%%%%%%%%%%%%%%%%

% Definition neuer Farben
	\definecolor{myblue}{rgb}{0.13,0.13,1}
	\definecolor{mygreen}{rgb}{0,0.5,0}
	\definecolor{myred}{rgb}{0.9,0,0}
	\definecolor{mygrey}{rgb}{0.46,0.45,0.48}

% Ein-/Ausblenden von Teilen des Dokuments
	\newcommand{\enableFig}[1]{#1}
	% entferne #1 aus {}, um Bilder in \enableFig-Umgebungen auszublenden
	\newcommand{\enableText}[1]{#1}	% speziell für Texte
	\newcommand{\enable}[1]{#1}	% allgemein
	\newcommand{\kommentar}[1]{\textcolor{mygrey}{#1}}
	
% Definitions- und Aussagenumgebung
	\newtheorem{thm}{Theorem}
	\newtheorem{definition}[thm]{Definition}
	\newtheorem{satz}[thm]{Satz}
	\newtheorem{lem}[thm]{Lemma}
	\newtheorem{kor}[thm]{%
		\iflanguage{german}{Korollar}{%
        \iflanguage{english}{Corollary}{}
        }%
    }
	\newtheorem{beob}[thm]{%
		\iflanguage{german}{Beobachtung}{%
        \iflanguage{english}{Observation}{}
        }%
	}
	\newtheorem{eig}[thm]{%
		\iflanguage{german}{Eigenschaft}{%
        \iflanguage{english}{Property}{}
    	}%	
	}
	\newtheorem{alg}[thm]{%
		\iflanguage{german}{Algorithmus}{%
        \iflanguage{english}{Algorithm}{}
    	}%	
	}

% Matematische Symbole
	\newcommand{\nats}{\mathbb{N}}
	\newcommand{\ints}{\mathbb{Z}}
	\newcommand{\rats}{\mathbb{Q}}
	\newcommand{\reals}{\mathbb{R}}
	\newcommand{\cplx}{\mathbb{C}}
	\renewcommand{\O}{\mathcal{O}}
	
	\renewcommand*\d{\mathop{}\!\mathrm{d}}
	\newcommand*\D[1]{\mathop{}\!\mathrm{d^#1}}
	
% weitere Befehle
	\newcommand{\TODO}{\Large\textcolor{red}{???}}

%%%%%%%%%%%%%%%%%%%%%%%%%%%%%%%%%%%%%%%%%%%%%%%%%%%%%%%%%%%%%%%%%%%%%%%%%%%%%%
%	N E U E  U M G E B U N G E N
%%%%%%%%%%%%%%%%%%%%%%%%%%%%%%%%%%%%%%%%%%%%%%%%%%%%%%%%%%%%%%%%%%%%%%%%%%%%%%

% Programmcode
	\lstnewenvironment{code}[1][]{\lstset{#1,
		backgroundcolor=\color{gray!20},
		inputencoding=utf8,
		extendedchars=true,
		literate={ä}{{\"a}}1 {Ä}{{\"A}}1 {ö}{{\"o}}1 {Ö}{{\"O}}1 {ü}{{\"u}}1 {Ü}{{\"U}}1 {ß}{{\ss}}1 {°}{{\textdegree}}{1} {\%}{\%}{1}
%		frame=single,                     % Rahmen um den Code
%    	rulecolor=\color{black},          % Rahmenfarbe (optional)
%	    framerule=0.5pt,                   % Rahmenstärke
%	    xleftmargin=0pt,                  % Linker Rand (optional)
%	    xrightmargin=0pt,                 % Rechter Rand (optional)
%	    aboveskip=10pt,                   % Abstand oberhalb
%	    belowskip=10pt 
	}
	}{}


	
	\selectlanguage{german}
	
\begin{document}
\title{Bruchrechnen}
\maketitle

\thispagestyle{empty}

\abstract{I am fucken schtupit}

\tableofcontents

\newpage

\section{Addition und Subtraktion}
Bei der Addition muss der Nenner gleich sein, dann werden die Zähler addiert (Erreichbar über das kleinste gemeinsame vielfache).
\[
	\frac{x}{a} + \frac{y}{b} = \frac{xb}{ab} + \frac{ya}{ab} = \frac{xb+ya}{ab}
\]
Bei der Subtraktion passiert genau das gleiche, nur dass man die Zähler subtrahiert.
\[
	\frac{x}{a} - \frac{y}{b} = \frac{xb}{ab} - \frac{ya}{ab} = \frac{xb-ya}{ab}
\]

\section{Brüche}
\subsection{Brüche kürzen}
Alle Brüche können mit dem größten gemeinsamen Teiler geteilt werden (Zähler und Nenner werden gleichermaßen geteilt).
\[
	\frac{2c}{4d} = \frac{\frac{2c}{2}}{\frac{4d}{2}} = \frac{c}{2d}
\]

\subsection{Brüche Addieren}
Beim addieren und subtrahieren müssen die Nenner gleich sein.
\[
	\frac{a}{b} + \frac{c}{b} = \frac{a+c}{b}
\]
\[
	\frac{a}{b} - \frac{c}{b} = \frac{a-c}{b}
\]
\subsection{Brüche mit Ganzzahlen verrechnen}
Ganzzahlen, die mit Brüchen multipliziert werden können in den Zähler verschoben werden.
\[
	\frac{a}{b} \cdot c = \frac{ac}{b}
\]

\subsection{Brüche multiplizieren / dividieren}
Beim multiplizieren werden alle Zähler und Nenner miteinander multipliziert.
\[
	\frac{x}{a} \cdot \frac{y}{b} = \frac{xy}{ab}
\]
Beim Dividieren wird vom zweiten Bruch der Kehrwert genommen und dann werden die Brüche miteineander multipliziert.
\[
	\frac{x}{a} \div \frac{y}{b} = \frac{x}{a} \div \frac{b}{y} = \frac{xb}{ay}
\]
Es gibt zwei häufige Schreibweisen:
\[
	\frac{\frac{x}{a}}{\frac{y}{b}} = \frac{x}{a} \div \frac{y}{b}
\]

\subsection{Brüche und Exponenten}
\[
	\frac{a}{b^{y}} = a \cdot b^{-y}
\]
Somit
\[
	ca^{-y} = \frac{c}{a^{y}}
\]

\subsection{Vorgehensweisen beim Kürzen}
Am besten Wurzeln im Nenner vermeiden.\\
\section{Potenzen}
\subsection{Potenzen multiplizieren}
Exponenten mit gleicher Basis
\[a^{b} \cdot a^{c} = a^{b + c}\]
Basen mit gleichem Exponenten
\[b^{a} \cdot c^{a} = (bc)^{a}\]
\subsection{Potenzen in Brüchen}
\[\frac{a^{b}}{a^{c}} = a^{b-c}\]
\section{Induktionsbeweise}
\subsection*{Aufbau eines Induktionsbeweises}
Zuerst formuliert man eine Aussage, die bewiesen werden soll. Zum Beispiel:
\begin{satz}
Für alle $n \in \mathbb{N}$ gilt
\[
\sum_{k=0}^{n}(2k-1) = n^{2} \enspace.
\]
\end{satz}
\begin{proof} Der Beweis wird per Induktion geführt.\\\\
\textbf{Induktionsanfang}
Zeige die Aussage für $n=1$.\\
\textbf{Induktionsvoraussetzung}
Nimm an, dass die Aussage für $n$ gilt.\\
\textbf{Induktionsschritt}
Zeige, dass die Aussage für $n+1$ gilt unter der Voraussetzung, dass sie für $n$ gilt.\\
Annahme: Die Summe aller ungeraden Zahlen ist eine Quadratzahl.\\
Diese Annahme wird mit irgendeiner Zahl überprüft (erster Dominostein)
\[A(1) = 1^{2} = 1\]
\[A(n) = n^{2}\]
\[A(n) = 1+3+5+ ... + (2n-1) = n^{2}\]
\[A(n+1) = 1+3+5+ ... + (2n-1) + (2(n+1)-1) = (n+1)^{2}\]
Beides zusammen
\[A(n+1) = n^{2} + (2(n+1)-1) = n^{2} + (2n+1) = n^{2} + 2n + 1 = (n+1)^{2}\]
Die Annahme
\[A(n) = n^{2}\]
und
\[A(n+1) = (n+1)^{2}\]
ist somit bewiesen.
\end{proof}

\section{Wichtige Regeln}
\subsection{Binomische Formeln}
\[(a+b)^{2} = a^{2} + 2ab + b^{2}\]
\[(a-b)^{2} = a^{2} - 2ab + b^{2}\]
\[(a-b)\cdot(a+b) = a^{2}-b^{2}\]
\end{document}	