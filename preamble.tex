\author{Benno Schörmann}
% Übersicht:
%	1. Allgemeine Einstellungen
%	2. Nützliche Pakete
%	3. Neue Befehle
%	4. Neue Umgebungen / Gestaltung

%%%%%%%%%%%%%%%%%%%%%%%%%%%%%%%%%%%%%%%%%%%%%%%%%%%%%%%%%%%%%%%%%%%%%%%%%%%%%%
%	A L L G E M E I N E  E I N S T E L L U N G E N
%%%%%%%%%%%%%%%%%%%%%%%%%%%%%%%%%%%%%%%%%%%%%%%%%%%%%%%%%%%%%%%%%%%%%%%%%%%%%%

% Spracheinstellungen und Kodierung
\usepackage[utf8]{inputenc}	% Eingabecodierung der Quelldatei
\usepackage[T1]{fontenc}	% Zeichenkodierung der verwendeten Schriftarten
\usepackage[ngerman]{babel}	% richtige Sprach- und Silbentrennung

% Seitenlayout
\usepackage[inner=2.5cm,outer=2.5cm,top=2cm,bottom=2cm]{geometry}

% Mikrotypographie für schöneres Schriftbild
\usepackage{microtype}

%%%%%%%%%%%%%%%%%%%%%%%%%%%%%%%%%%%%%%%%%%%%%%%%%%%%%%%%%%%%%%%%%%%%%%%%%%%%%%
%	N Ü T Z L I C H E  P A K E T E
%%%%%%%%%%%%%%%%%%%%%%%%%%%%%%%%%%%%%%%%%%%%%%%%%%%%%%%%%%%%%%%%%%%%%%%%%%%%%%

% Mathekram
\usepackage{amsmath, amsthm, amsfonts, amssymb}

% Tikz
\usepackage{tikz}
\usetikzlibrary{
	arrows,
	arrows.meta,
	automata,
	calc,
	decorations.pathmorphing,
	fadings,
	fit,
	positioning,
	shapes
}
\usepackage{pgfplots}

% Verweise
\usepackage{nameref}

% Links
\usepackage{hyperref}
\hypersetup{
	colorlinks=true,
	linkcolor={black},
	citecolor={black},
	urlcolor={myblue},
	hyperindex=true
}

% Farben
\usepackage{xcolor}

% Aufzählungen
\usepackage{enumitem}

% Tabellen
\usepackage{array, multirow}

% Bedingte Anweisungen und Logik
\usepackage{ifthen, pgffor}

% Programmcodeumgebung
\usepackage{listings}
\lstset{language=Java}

% Schöne Anführungszeichen
\usepackage{csquotes}

%%%%%%%%%%%%%%%%%%%%%%%%%%%%%%%%%%%%%%%%%%%%%%%%%%%%%%%%%%%%%%%%%%%%%%%%%%%%%%
%	N E U E  B E F E H L E
%%%%%%%%%%%%%%%%%%%%%%%%%%%%%%%%%%%%%%%%%%%%%%%%%%%%%%%%%%%%%%%%%%%%%%%%%%%%%%

% Farben
\definecolor{myblue}{rgb}{0.13,0.13,1}
\definecolor{mygreen}{rgb}{0,0.5,0}
\definecolor{myred}{rgb}{0.9,0,0}
\definecolor{mygrey}{rgb}{0.46,0.45,0.48}

% Ein-/Ausblenden von Teilen des Dokuments
\newcommand{\enableFig}[1]{#1}
\newcommand{\enableText}[1]{#1}
\newcommand{\enable}[1]{#1}
\newcommand{\kommentar}[1]{\textcolor{mygrey}{#1}}

% Theorem-Umgebungen
\theoremstyle{definition}
\newtheorem{definition}{Definition}[section]

\theoremstyle{plain}
\newtheorem{satz}[definition]{Satz}
\newtheorem{thm}[definition]{Theorem}
\newtheorem{lem}[definition]{Lemma}
\newtheorem{kor}[definition]{Korollar}

\theoremstyle{remark}
\newtheorem{beob}[definition]{Beobachtung}
\newtheorem{eig}[definition]{Eigenschaft}
\newtheorem{alg}[definition]{Algorithmus}

% Mathematische Symbole
\newcommand{\nats}{\mathbb{N}}
\newcommand{\ints}{\mathbb{Z}}
\newcommand{\rats}{\mathbb{Q}}
\newcommand{\reals}{\mathbb{R}}
\newcommand{\cplx}{\mathbb{C}}
\renewcommand{\O}{\mathcal{O}}

\renewcommand*\d{\mathop{}\!\mathrm{d}}
\newcommand*\D[1]{\mathop{}\!\mathrm{d^#1}}

% TODO
\newcommand{\TODO}{\Large\textcolor{red}{???}}

%%%%%%%%%%%%%%%%%%%%%%%%%%%%%%%%%%%%%%%%%%%%%%%%%%%%%%%%%%%%%%%%%%%%%%%%%%%%%%
%	N E U E  U M G E B U N G E N
%%%%%%%%%%%%%%%%%%%%%%%%%%%%%%%%%%%%%%%%%%%%%%%%%%%%%%%%%%%%%%%%%%%%%%%%%%%%%%

% Programmcode
\lstnewenvironment{code}[1][]{
	\lstset{#1,
		backgroundcolor=\color{gray!20},
		inputencoding=utf8,
		extendedchars=true,
		literate={ä}{{\"a}}1 {Ä}{{\"A}}1 {ö}{{\"o}}1 {Ö}{{\"O}}1 {ü}{{\"u}}1 {Ü}{{\"U}}1 {ß}{{\ss}}1 {°}{{\textdegree}}{1} {\%}{\%}{1}
	}
}{}
