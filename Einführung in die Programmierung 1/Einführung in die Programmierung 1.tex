\documentclass[11pt, a4paper]{article}

	\author{Benno Schörmann}
% Übersicht:
%	1. Allgemeine Einstellungen
%	2. Nützliche Pakete
%	3. Neue Befehle
%	4. Gestaltung des Blattes

%%%%%%%%%%%%%%%%%%%%%%%%%%%%%%%%%%%%%%%%%%%%%%%%%%%%%%%%%%%%%%%%%%%%%%%%%%%%%%
%	A L L G E M E I N E  E I N S T E L L U N G E N
%%%%%%%%%%%%%%%%%%%%%%%%%%%%%%%%%%%%%%%%%%%%%%%%%%%%%%%%%%%%%%%%%%%%%%%%%%%%%%

% Spracheinstellungen und Kodierung
	\usepackage[ngerman]{babel}	% richtige Sprach- und Silbentrennung
	\usepackage[utf8]{inputenc}	% Eingabecodierung der Quelldatei
	\usepackage[T1]{fontenc}	% Zeichenkodierung der verwendeten Schriftarten
	
% Seitenlayout
	\usepackage[inner=2.5cm,outer=2.5cm,top=2cm,bottom=2cm]{geometry}

%%%%%%%%%%%%%%%%%%%%%%%%%%%%%%%%%%%%%%%%%%%%%%%%%%%%%%%%%%%%%%%%%%%%%%%%%%%%%%
%	N Ü T Z L I C H E  P A K E T E
%%%%%%%%%%%%%%%%%%%%%%%%%%%%%%%%%%%%%%%%%%%%%%%%%%%%%%%%%%%%%%%%%%%%%%%%%%%%%%

% Mathekram
	\usepackage{amsmath}	% Umgebungen und Funktionen für Gleichungsdarstellung
	\usepackage{amsthm}		% mathematische Umgebungen, wie Theorem, Sätze, Korollare,
	\usepackage{amsfonts}	% mathematische Schriftarten, z.B. für Integer oder Reals
	\usepackage{amssymb}	% weitere mathematische Symbole
	
% Tikz
	\usepackage{tikz}
	\usetikzlibrary{
		arrows,
		arrows.meta,
		automata,
		calc,
		decorations.pathmorphing,
		fadings,
		fit,
		positioning,
		shapes
	}
	\usepackage{pgfplots}

% Verweise
	\usepackage{nameref}

% Links
	\usepackage{hyperref}	% ermöglicht interne und externe Links
	\hypersetup{%
		colorlinks=true,
		%frenchlinks=true,
		linkcolor={black},
		%linkbordercolor = white,
	    citecolor={black},
	    urlcolor={myblue},
	    hyperindex=true
	}

% Farben
	\usepackage{xcolor}		% Verwaltung von Farben
	
% Aufzählungen
	\usepackage{enumitem}
	
% Tabellen
	\usepackage{array}
	\usepackage{multirow}
	
% Bedingte Anweisungen und Logik
	\usepackage{ifthen}
	\usepackage{pgffor}		% Schleifen
	
% Programmcodeumgebung
	\usepackage{listings}
	\lstset{language=Java}
	% Einstellungen beim Punkt "Gestaltung des Blattes"

%%%%%%%%%%%%%%%%%%%%%%%%%%%%%%%%%%%%%%%%%%%%%%%%%%%%%%%%%%%%%%%%%%%%%%%%%%%%%%
%	N E U E  B E F E H L E
%%%%%%%%%%%%%%%%%%%%%%%%%%%%%%%%%%%%%%%%%%%%%%%%%%%%%%%%%%%%%%%%%%%%%%%%%%%%%%

% Definition neuer Farben
	\definecolor{myblue}{rgb}{0.13,0.13,1}
	\definecolor{mygreen}{rgb}{0,0.5,0}
	\definecolor{myred}{rgb}{0.9,0,0}
	\definecolor{mygrey}{rgb}{0.46,0.45,0.48}

% Ein-/Ausblenden von Teilen des Dokuments
	\newcommand{\enableFig}[1]{#1}
	% entferne #1 aus {}, um Bilder in \enableFig-Umgebungen auszublenden
	\newcommand{\enableText}[1]{#1}	% speziell für Texte
	\newcommand{\enable}[1]{#1}	% allgemein
	\newcommand{\kommentar}[1]{\textcolor{mygrey}{#1}}
	
% Definitions- und Aussagenumgebung
	\newtheorem{thm}{Theorem}
	\newtheorem{definition}[thm]{Definition}
	\newtheorem{satz}[thm]{Satz}
	\newtheorem{lem}[thm]{Lemma}
	\newtheorem{kor}[thm]{%
		\iflanguage{german}{Korollar}{%
        \iflanguage{english}{Corollary}{}
        }%
    }
	\newtheorem{beob}[thm]{%
		\iflanguage{german}{Beobachtung}{%
        \iflanguage{english}{Observation}{}
        }%
	}
	\newtheorem{eig}[thm]{%
		\iflanguage{german}{Eigenschaft}{%
        \iflanguage{english}{Property}{}
    	}%	
	}
	\newtheorem{alg}[thm]{%
		\iflanguage{german}{Algorithmus}{%
        \iflanguage{english}{Algorithm}{}
    	}%	
	}

% Matematische Symbole
	\newcommand{\nats}{\mathbb{N}}
	\newcommand{\ints}{\mathbb{Z}}
	\newcommand{\rats}{\mathbb{Q}}
	\newcommand{\reals}{\mathbb{R}}
	\newcommand{\cplx}{\mathbb{C}}
	\renewcommand{\O}{\mathcal{O}}
	
	\renewcommand*\d{\mathop{}\!\mathrm{d}}
	\newcommand*\D[1]{\mathop{}\!\mathrm{d^#1}}
	
% weitere Befehle
	\newcommand{\TODO}{\Large\textcolor{red}{???}}

%%%%%%%%%%%%%%%%%%%%%%%%%%%%%%%%%%%%%%%%%%%%%%%%%%%%%%%%%%%%%%%%%%%%%%%%%%%%%%
%	N E U E  U M G E B U N G E N
%%%%%%%%%%%%%%%%%%%%%%%%%%%%%%%%%%%%%%%%%%%%%%%%%%%%%%%%%%%%%%%%%%%%%%%%%%%%%%

% Programmcode
	\lstnewenvironment{code}[1][]{\lstset{#1,
		backgroundcolor=\color{gray!20},
		inputencoding=utf8,
		extendedchars=true,
		literate={ä}{{\"a}}1 {Ä}{{\"A}}1 {ö}{{\"o}}1 {Ö}{{\"O}}1 {ü}{{\"u}}1 {Ü}{{\"U}}1 {ß}{{\ss}}1 {°}{{\textdegree}}{1} {\%}{\%}{1}
%		frame=single,                     % Rahmen um den Code
%    	rulecolor=\color{black},          % Rahmenfarbe (optional)
%	    framerule=0.5pt,                   % Rahmenstärke
%	    xleftmargin=0pt,                  % Linker Rand (optional)
%	    xrightmargin=0pt,                 % Rechter Rand (optional)
%	    aboveskip=10pt,                   % Abstand oberhalb
%	    belowskip=10pt 
	}
	}{}


	
	\selectlanguage{german}
	
\begin{document}

\title{Einführung in die Programmierung}
\maketitle

\thispagestyle{empty}

\abstract{Vorl. Frau Prof. Dr. Judith Jakob}

\newpage

\tableofcontents

\newpage

\section{Tipps}

\subsection{Für Klausuren lernen}
\begin{enumerate}
\item Vorlesung besuchen
\item VL zuhause durchgehen bis alles verstanden wurde (Nacharbeiten)
\item Übung zuhause vorbereiten
\item Übung nur fürs abgleichen der Lösungen
\item Praktikum unvirbereitet rein
\item Immer bis zum Ende
\item Autocompletion deaktivieren
\end{enumerate}

\subsection{Klausurphase}
Anki nutzen\\Spicker\\Spicker während dem Üben schreiben\\Karteikarten erstellen\\
\\Lernmethode: Vorlesung angucken -> Übung lösen + Spicker erstellen\\Selber kontrollieren\\
\\Probenklausur: Papier + Timer (So realistisch wie möglich)\\
\\So viele Probeklausuren wie möglich machen

\subsection{Weiteres}
Bonuspunktetests definitiv machen (Kann Klausur um eine ganze Note verbessern)\\
\\100 Stunden programmieren für Klausur\\
\\UML Diagramme einfach auswendig lernen\\99 prozent Schema f\\
\\10 prozent Wissensfragen\\90 prozent Algorithmen aus Übung\\Selten Probleme lösen

\subsection{KI}
Keine Lösungen generieren\\Vorlesungen zusammen fassen lassen

\subsection{Modulmenge}
Anstatt 6 Module eventuell 5 Module nehmen und die richtig machen

\section{Organisatorisches}
\subsection{Bonuspunktetests}
Test 1: 25.11. bis 28.11. in den ÜPP zu den Themen der Vorlesung 1 bis 6
Test 1: 20.01. bis 23.01. in den ÜPP zu den Themen der Vorlesung 6 bis 12
\subsection{Projektwoche}
Projektwoche nur bestanden / nicht bestanden\\\\Anmeldefrist 03.11.-15.11.\\Januar: 26.01.2026 - 30.01.2026
\section{Vorlesung 1}
\subsection{Definitionen}
Info-matik -> Zusammengesetzt aus Informationen und Automatismus\\
\subsubsection{Binary bla bla}
Computer arbeiten in multiplen von Bytes (8, 16, 32, 64...)\\Kleinste addressierbare Einheit ist ein Byte\\Register sind "Unwichtig" da wir mit abstrakten Programmiersprachen arbeiten werden
\subsection{Informationen und Daten}
Informationen werden repräsentiert als Daten\\Daten werden interpretiert um Informationen zu bekommen
\subsection{Zahlensysteme}
Alle genutzten Systeme sind Stellenwertsysteme, also sind die Positionen der einzelnen Zahlen wichtig\\
\[
	\sum_{k=0}^{N}(b_{k} \cdot 10^{k}), b_{k} \in \mathbb{N} \{0,..., N-1\}
\]
Wobei $k$ der Index ist und $b_{k}$ die Zahl an dem jeweiligen Index ist\\\\Man nehme $1A7$\\
\[
	\sum_{k=0}^{2}
\]

\subsection{Umwandlung von Dezimal ganzzahlen in Binär}
Modulorechnen
Beispiel n = 43:\\
\[43/2 = 21 R1\]
\[21/2 = 10 R1\]
\[10/2 = 5 R0\]
\[5/2 = 2 R1\]
\[2/2 = 1 R0\]
\[1/2 = 0 R1\]
Bei 0 stoppen, dann von 0 nach oben Reste aufschereiben
\[(43)_{10} = (101011)_{2}\]

\subsection{Umwandlung von Dezimal kommazahlen in Binär}
0,1 zu Binär:\\
Die Kommazahl mal zwei nehmen und sobald 1, da steht wird der Rest genommen und mit 0, weiter gemacht
\[0,1 \cdot 2 = 0,2\]
\[0,2 \cdot 2 = 0.4\]
\[0,4 \cdot 2 = 0,8\]
\[0,8 \cdot 2 = 1,6\]
\[0,6 \cdot 2 = 1,2\]
\[0,2 \cdot 2 = 0,4\]
\[0,4 \cdot 2 = 0,8\]

\[(0,1)_{10} = (0,00011\overline{0011})_{2}\]

\end{document}